\documentclass[11pt,preprint]{aastex}
%\documentclass{emulateapj}
\usepackage{url}
%\usepackage{natbib}
%\usepackage{xspace}
\def\arcsec{$^{\prime\prime}$}
\bibliographystyle{apj}
\newcommand\degree{{^\circ}}
\newcommand\surfb{$\mathrm{mag}/\square$\arcsec}
\newcommand\Gyr{\rm{~Gyr}}
\newcommand\msun{\rm{M}_\odot}
\newcommand\kms{km s$^{-1}$}
\newcommand\al{$\alpha$}
\newcommand\ha{$\rm{H}\alpha$}
\newcommand\hb{$\rm{H}\beta$}
\usepackage{graphicx}
\usepackage{subfigure}

\begin{document}

\title{Attempting to Understand Power Spectra}



\section{Basic Procedure}

We take an image and zero-pad the borders.  We then use scipy to compute a 2D FFT and shift the output so the zero-frequency components (i.e., large spatial scales) are in the center.  We calculate the 2D power spectrum as the square of the absolute value of the 2D-FFT.  Finally, we azimuthally average the 2D power spectrum to obtain a 1D power spectrum.


\section{Simple Images}

Here we generate simple images of the form
\begin{equation}
z = sin(2 \pi x/s)sin(2 \pi y/s)
\end{equation}
where $s$ is a variable scale factor.  Results for $s=$ 500, 10, and 5 are shown in Figures~\ref{s500}, \ref{s10}, and \ref{s5} respectively.


\begin{figure}
\plotone{sin_ps500.png}
\caption{Simple sin structure in both x and y.  As expected, the power spectrum peaks at small $k$. \label{s500}}
\end{figure}

\begin{figure}
\plotone{sin_ps10.png}
\caption{Simple sin structure in both x and y.  Now that the structure is smaller scale, the power spectrum peak moves to larger $k$. \label{s10}}
\end{figure}


\begin{figure}
\plotone{sin_ps5.png}
\caption{Simple sin structure in both x and y.  Now that the structure is smaller scale, the power spectrum peak moves to larger $k$. \label{s5}}
\end{figure}


\section{Cloud Images}

Clouds generated by the French group are shown in Figure~\ref{oldc}, while clouds based on an image are shown in Figure~\ref{newc}.  The 1D power spectra look fairly similar, but there is interesting structure in the 2D power spectra for the French clouds that seems to make the final cloud image look unrealistic.

\begin{figure}
\plotone{oldclouds.png}
\caption{Our old cloud code.  Image of the clouds upper left, the real part of the FFT on the upper right, the 2D power spectrum on the lower left, and the azimuthally averaged power spectrum on the lower right. \label{oldc}}
\end{figure}


\begin{figure}
\plotone{newclouds.png}
\caption{Our newer cloud code. Panels the same as Figure~\ref{oldc}\label{newc}}
\end{figure}




\section{Outstanding Questions}
\begin{enumerate}
\item{Why does the power peak at $k=0$ in Figure~\ref{newc}?  It looks like the cloud structure has a typical size, so why isn't there a peak there?}
\item{Given a 1D power spectrum, what is the best way to then construct a 2D power spectrum (and then inverse-FFT and create an image)?}
\item{How can one go gracefully go from doing regular FFTs on a tangent plane-projection to using full spherical harmonics on the sky?}
\end{enumerate}




\end{document}

% LocalWords:  scipy FFT ps png oldclouds azimuthally newclouds FFTs
